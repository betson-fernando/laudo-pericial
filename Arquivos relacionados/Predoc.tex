\RequirePackage[2020-02-02]{latexrelease}
\usepackage[brazil]{babel}
\usepackage[utf8]{inputenc}
\usepackage{setspace}
\usepackage{graphicx}
\usepackage{float}

\usepackage[a4paper,left=30mm,top=30mm,bottom=20mm,right=20mm,showframe]{geometry}
\usepackage{titlesec}

\usepackage{fancyhdr}
\usepackage{helvet}
\renewcommand{\familydefault}{\sfdefault}
\usepackage[small,bf,hang,justification=justified]{caption}   TODO: ainda não testei a linha acima, só quando colocar figuras.
\usepackage{calc}
\usepackage{enumerate}
\usepackage{indentfirst}


\usepackage{etoolbox} TODO: se possível, excluir este pacote. Tentar ficar livre do newrobustcmd

\usepackage{xcolor}
\usepackage{xwatermark}

\usepackage{lmodern}% %\usepackage{textcomp}% %\usepackage{lastpage}% %\usepackage[utf8]{inputenc} %\usepackage[T1]{fontenc}

\graphicspath{{../Fotografias/}{../}}          % Buscar figuras nos diretórios selecionados

\titleformat*{\section}{\bf}  %.......................................................................................................................................Alterar formato das sessões para tamanho do texto, negrito.
\titleformat*{\subsection}{\bf}  %................................................................................................................................Alterar formato das subssessões para tamanho do texto, negrito
\titlelabel{\thetitle.\quad}  %........................................................................................................................................Formato do título é "X) Nome"


%\usepackage{etoolbox}  %......................................................................................................................................Apenas para newrobustcmd e ifnumcomp. GARANTIR QUE NÃO TEM MAIS UTILIDADE
%\usepackage{anysize}                                % Descobrir para que serve esta package
%\usepackage{layouts}                                         % Garantir que esta package não tem mais utilidade.
%\usepackage[svgnames]{xcolor}				        %  Permite alterar as cores do texto (usado somente pra gambiarra de contar figuras, no final do documento)
%\usepackage{refcount}					       % Permite transformar referências em contadores. GARANTIR QUE NÃO TEM MAIS UTILIDADE PARA REFERÊNCIAS E CONTADORES
%\geometry{a4paper, total={160mm,222.65mm}, left=30mm, top=10mm}           % Define tamanho do texto e margens. SUBTITUÍDO PELA DEFINIÇÃO DE GEOMETRY
%\captionsetup{format=hang, justification=justified}                                              % Alinha as linhas da legenda. LINHA EXCLUÍDA. INFORMAÇÕES DIRETO NA DEFINIÇÃO DE CAPTION


\setlength{\abovecaptionskip}{2pt}							 % Ajusta separação figura-legenda
\setlength{\parindent}{1.5cm}                                                                                % Configura identação.



%Macro para adicionar figura: \f{path.jpg}{legenda}
\newcommand{\f}[2]{
    \begin{figure}[H]
        \centering
        \includegraphics[height=9 cm]{#1}
        \settowidth{\imgwidth}{\includegraphics[height=9 cm]{#1}}
        \captionsetup{width=\imgwidth}
        \caption{#2}
        \label{#1}
    \end{figure}
}

\newcommand{\f}[2]{\fig{#1}{#2}{#1}}
\newcommand{\iml}{Instituto de Medicina Legal Antônio Persivo Cunha (IMLAPC)}
\newcommand{\igfec}{Instituto de Genética Forense Eduardo Campos (IGFEC) para exame comparação do perfil genético potencialmente presente na(s) amostra(s) com os perfis
genéticos de suspeitos que venham a ser elencadospela delegacia responsável pela investigação da ocorrência. Detalhes desta coleta estão descritos no Requerimento de Perícia ao
IGFEC, cuja cópia será enviada, para conhecimento da investigação, no mesmo processo SEI que este Laudo Pericial}
\newcommand{\bal}{Setor de Balística forense do Instituto de Criminalística Professor Armando Samico e, caso se fizer necessária perícia especializada, a Autoridade Policial deverá
encaminhar ofício àquele setor elencando os questionamentos pertinentes.}


\newlength{\imgwidth}                     % Para alinhar legendas às imagens
\newcounter{pgf}			    % Contador para referenciar a página final
\newcounter{reffig}


%============================CONSTRUÇÃO DO CABEÇALHO==========================
\fancyhf{}                                                                           % Retirar régua do cabeçalho
\renewcommand{\headrulewidth}{0pt}
\pagestyle{fancy}
\voffset=-18.4mm
\topmargin=0mm
\headheight=37.5mm
\headsep=3mm

\lhead{}
\chead{\includegraphics[width=15.5cm]{Cab.png}}
\rhead{}

%============================= CONSTRUÇÃO DO RODAPÉ===========================
\footskip=7.75mm

\cfoot{\fontsize{10}{0} \selectfont {\sl{Laudo Pericial nº \caso{ }- REP nº \rep} \hfill {Página \thepage}}\\\rule{16cm}{2pt}  \\\baselineskip=12pt\bf Rua Doutor João Lacerda, nº 395, bairro do Cordeiro, Recife/ PE – CEP: 50.711-280 \newline Administrativo/ Plantão: (81) 3184-3547 - E-mail: geph.dhpp@gmail.com}
%=========================================================================

\newwatermark[allpages,ypos=-55pt]{\includegraphics{Marcadagua.png}}           % Marca d'água